\def\ptsz{11pt}
\documentclass[\ptsz]{article}

%%%%%%%%%%%%%%%%%%%%%%%%%%%%%%%%%%%%%%%%%%
%%  Packages
%\def\usetikz#1{}
\def\usetikz#1{#1}
\usepackage{amssymb,amsmath}
\usepackage{graphicx}  %needed for the includegraphics command
\usepackage{verbatim} % needed for comment environment
\usepackage{ifthen} % needed for ifthenelse
\usepackage{hyperref} %for comments
\usepackage{xcolor} %for comments
\usepackage{enumerate}% for lettering equations
\usetikz{\usepackage{tikz}
\usepgflibrary{arrows}
\usetikzlibrary{calc,through,backgrounds} %****no spaces after commas***
\usetikzlibrary{decorations.pathmorphing}
\usetikzlibrary{decorations.markings}
\usetikzlibrary{positioning} %for below=of xxx syntax in nodes
\usetikzlibrary{backgrounds,fit} %for background layers example
}
\usepackage[all,arc]{xy} %load this one last
%%%%%%%%%%%%%%%%%%%%%%%%%%%%%%%%%%%%%%%%%%

%%%%%%%%%%%%%%%%%%%%%%%%%%%%%%%%%%%%%%%%%%
%%  Settings

\textwidth=6in
\textheight=8.9in
\topmargin=-0.5in
\evensidemargin=0in
\oddsidemargin=0in
\parindent=0pt
\parskip=1ex

\unitlength =  10.0 pt 
%%%%%%%%%%%%%%%%%%%%%%%%%%%%%%%%%%%%%%%%%%
%%  Directories
\def\jpgdir{../img}
\def\imgdir{../img}

%%%%%%%%%%%%%%%%%%%%%%%%%%%%%%%%%%%%%%%%%%
%%%%%%%%%%%%%%%%%%%%%%%%%%%%%%%%%%%%%%%%%%
%% Version 5
%    Jan 3, 2013
%       Added \subhead, \problem, \numexamp
%    Feb 27, 2012
%       Added \xbar, \Var, \Cov  macros
%    Feb 11, 2012
%       Reorganized to use with header.tex and new mkpdf.sh
%       added some macros
%    Feb 5, 2012
%       Copied from 18.03
%       Added myhide code.
%       Removed topics codes
%%%%%%%%%%%%%%%%%%%%%%%%%%%%%%%%%%%%%%%%%%

%%  Macros
\def\partHeading#1#2{\mcent{\textrm{\large \bf Part #1 \hspace{2.3ex}#2}}}


%%  Simple macros
\def\mycomment#1{{\color{red}#1}}
\def\mcent#1{\hspace*{\stretch{1}}#1\hspace*{\stretch{1}}{ }}
\def\hs#1{\hspace*{#1 ex}}
\def\nl#1{\\[.#1ex]}
\def\ds{\displaystyle}
\def\cont{\vspace*{\stretch{1}}\emph{(continued)} \newpage}
\def\reading{\textbf{Reading. }}
\def\examples{\textbf{Examples. }}
\def\example{\textbf{Example. }}
\def\examp#1{\textbf{Example #1. }}
\def\problem{\textbf{Problem. }}
\def\problems{\textbf{Problems. }}
\def\subhead#1{\textbf{#1.}}
\def\indetop#1#2#3{\stackrel{\; \scriptstyle{{#1}{#3}{#2}}}{=} \;}
\def\indet#1#2{\stackrel{\scriptstyle{\; \frac{#1}{#2}}}{=} \;}
\def\indetzero{\indet{0}{0}}
\def\indetinf{\indetop{\infty}{\infty}{/}}
\def\qedbox{\rule{1ex}{1ex}}

\def\mybull{$\bullet$}
\def\W{\Omega}
\def\w{\omega}
\def\mycap{\,\cap\,}
\def\mycup{\,\cup\,}
\def\Norm{\text{N}}

\def\myref#1{(\ref*{#1})}

\def\mypart#1#2{\frac{\partial #1}{\partial #2}}
%      fully specified partial
\def\fp#1#2#3{(\partial #1/\partial #2)_{#3}}                %without \frac
\def\ffp#1#2#3{\left.\frac{\partial #1}{\partial #2}\right|_{#3}}  %with \frac
\def\mpp#1#2#3{\left(\mypart{#1}{#2}\right)_{#3}} 
\def\myderiv#1#2{\frac{d#1}{d#2}}
\def\mynderiv#1#2#3{\frac{d^{#3}#1}{d#2^{#3}}}
\def\mygrad{\boldsymbol{\nabla}}         %gradient
\def\gf#1#2{\left.\mygrad #1\right|_{#2}}
\def\endandindent{\\ \hspace*{20pt}}
\def\e#1{\mathrm{e}^{#1}}
\def\myIm{\textup{Im}}
\def\myRe{\textup{Re}}
\def\conj#1{\overline{#1}}
\def\tran{^\mathrm{T}}
\def\ans{{\bf \underline{answer:}}{ }}
\def\th{$^{\mathrm{th}}$}
\def\myhead#1{\noindent \textbf{#1}}
\def\tbf#1{\textbf{#1}}
\def\mysep{: }
\def\lap{{\mathcal{L}}}
\def\ilap{\lap^{-1}}
\def\myimply{\; \Rightarrow \;}
\def\myequiv{\; \Leftrightarrow \;}
\def\ft{\hat}
\def\xbar{\overline{X}}
\def\Var{\textup{Var}}
\def\Cov{\textup{Cov}}
\def\bypartshelp#1#2#3#4#5#6#7{\framebox{$\begin{array}[#7]{lll} 
#5=#1 & d#6=#2\\
d#5=#3 & #6=#4 \end{array}$}}
\def\byparts#1#2#3#4{\bypartshelp{#1}{#2}{#3}{#4}{u}{v}{l}}
\def\bypartst#1#2#3#4{\bypartshelp{#1}{#2}{#3}{#4}{u}{v}{t}}

%matlab 
\newcommand{\mlcmd}[2]{\textgreater{} \texttt{#1}\\ \texttt{#2}\\}
\newcommand{\mlcmdna}[1]{\textgreater{} \texttt{#1}\\}
\newcommand{\mlans}[2]{#1 =\\[.5ex]\hs3\parbox{5in}{#2}}
\def\mlcomm#1{{\color{blue}\% #1}\\}
\def\mlinstr#1{\mlcomm{#1}}
\def\mlcar{\^{ }}

%counters
\newcounter{examplectr}
\setcounter{examplectr}{1}
\renewcommand{\theexamplectr}{\arabic{examplectr}}
\newcommand{\numexamp}{\textbf{Example \arabic{examplectr}.{ }}\addtocounter{examplectr}{1}}

%matrices
\def\defleftbrace{(}
\def\defrightbrace{)}
\def\twobytwohelp#1#2#3#4#5#6#7{\left#6 \begin{array}{#1} #2 & #3 \\ #4 & #5 \end{array} \right#7}
\def\twobytwo#1#2#3#4{\twobytwohelp{rr}{#1}{#2}{#3}{#4}{\defleftbrace}{\defrightbrace}}
\def\twobytwoc#1#2#3#4{\twobytwohelp{cc}{#1}{#2}{#3}{#4}{\defleftbrace}{\defrightbrace}}
\def\twobytwodet#1#2#3#4{\twobytwohelp{rr}{#1}{#2}{#3}{#4}{|}{|}}
\def\twobytwodetc#1#2#3#4{\twobytwohelp{cc}{#1}{#2}{#3}{#4}{|}{|}}
\def\twobyonehelp#1#2#3{\left\defleftbrace \begin{array}{#1} #2\\ #3  \end{array} \right\defrightbrace}
\def\twobyone#1#2{\twobyonehelp{r}{#1}{#2}}
\def\twobyonec#1#2{\twobyonehelp{c}{#1}{#2}}
\def\threebythree#1#2#3#4#5#6#7#8#9{\left\defleftbrace \begin{array}{rrr} #1&#2&#3\\ #4&#5&#6\\ #7&#8&#9 \end{array} \right\defrightbrace}
\def\threebythreec#1#2#3#4#5#6#7#8#9{\left\defleftbrace \begin{array}{ccc} #1&#2&#3\\ #4&#5&#6\\ #7&#8&#9 \end{array} \right\defrightbrace}
\def\threebythreedet#1#2#3#4#5#6#7#8#9{\left| \begin{array}{rrr} #1&#2&#3\\ #4&#5&#6\\ #7&#8&#9 \end{array} \right|}
\def\threebythreedetc#1#2#3#4#5#6#7#8#9{\left| \begin{array}{ccc} #1&#2&#3\\ #4&#5&#6\\ #7&#8&#9 \end{array} \right|}
\def\threebyone#1#2#3{\left\defleftbrace \begin{array}{r} #1\\ #2\\ #3 \end{array} \right\defrightbrace}
\def\threebyonec#1#2#3{\left\defleftbrace \begin{array}{c} #1\\ #2\\ #3 \end{array} \right\defrightbrace}
\def\threebytwo#1#2#3#4#5#6{\left\defleftbrace \begin{array}{rr} #1&#2\\ #3&#4\\ #5&#6 \end{array} \right\defrightbrace}
\def\threebytwoc#1#2#3#4#5#6{\left\defleftbrace \begin{array}{cc} #1&#2\\ #3&#4\\ #5&#6 \end{array} \right\defrightbrace}

%Vectors and line segments
\def\vb#1{\mathbf{#1}}  %bold
\def\va#1{\overrightarrow{#1}}  %arraow
\def\vba#1{\overrightarrow{\mathbf{#1}}}  %bold/arraow
\def\vl#1{\overline{#1}} %overline
\def\vbl#1{\overline{\mathbf{#1}}} %bold/overline
\def\vu#1{\widehat{#1}} %unit vector
\def\vbu#1{\widehat{\mathbf{#1}}} %bold/unit vector
\def\un#1{\frac{#1}{\left|#1\right|}}  %normalize vector
\def\vbi{\vbu{i}}
\def\vbj{\vbu{j}}
\def\vbk{\vbu{k}}
\def\vc#1{\langle #1 \rangle}
\def\vcb#1{\left\langle #1 \right\rangle}

\newcommand{\st}[1]{\ensuremath{^{\scriptstyle \textrm{#1}}}}
\newcommand{\undertext}[1]{\ensuremath{\underline{\textrm{#1}}}}
\newcommand{\fracc}{\displaystyle\frac}
\newcommand{\summ}{\displaystyle\sum}

%%%%%%%%%%%%%%%%%%%%%%%%%%%%
%% tikz
\usetikz{\tikzset{
addarrow/.style={postaction={decorate},
        decoration={markings,mark=at position #1 with {\arrow{>}}}}
}}

%%%%%%%%%%%%%%%%%%%%%%%%%%%
%% Sample Complex formatting macros
%\makeatletter
%\renewcommand\section{\goodbreak\@startsection {section}{1}{\z@}%
%%%                                   {-3.5ex \@plus -1ex \@minus -.2ex}%
%                                   {-3.5ex \@plus -1ex \@minus -.2ex}%
%                                   {2.3ex \@plus.2ex}%
%                                   {\normalfont\large\bfseries%
%                                     \centering\sectionname\ }}

%\renewcommand\subsection{\goodbreak\@startsection{subsection}{2}{\z@}%
%%                                     {-3.25ex\@plus -1ex \@minus -.2ex}%
 %                                    {-2ex\@plus -1ex \@minus -.2ex}%
%%                                     {1.5ex \@plus .2ex}%
%                                     {1.25ex \@plus .2ex}%
%                                     {\normalfont\large\bfseries}}
%\makeatother

%\renewcommand{\thesection}{\Roman{section}}
%\newcommand\sectionname{Part}
%\newcommand{\alphalist}{% changes enumerate 1st level to (a)...(z)
%  \renewcommand{\theenumi}{\alph{enumi}}%
%  \renewcommand{\labelenumi}{\theenumi)}%
%}
%\alphalist
%%%%%%%%%%%%%%%%%%%%%%%%%%%%%%%%%%%%%%%%%%

\usepackage{ifthen} % needed for ifthenelse

\def\needstobedone{XXX***XXX}

\def\whichterm{Spring 2013}
%\def\whichprogram{ESG}
\def\teacher{Dr. Jeremy Orloff and Dr. Jonathan Bloom}
\def\email{jorloff}

%\def\website{web.mit.edu/jorloff/www/18.05}
%\def\matlabadd{\intentionalerror{need matlab dir}}
%\def\matlabdir{\intentionalerror{need matlab dir}}

\def\wheredue{in 2-285{}}
%\def\finalinfo{To be anounced}

\def\psdue#1{\textbf{Pset #1 due}}
%\def\exam#1#2{\textbf{Exam #1: covering #2}}
%\def\review{Discussion, review and catch up.}
%\def\reviewentry#1#2{\textbf{Continuation: } (#1) \, #2}
%\def\recitationentry#1#2{\textbf{Problem section:} (#1) \, #2}

%\def\examonecovers{\intentionalerror}
%\def\examtwocovers{\intentionalerror}

%Problem set due dates
\def\duetime{4:30 PM}
\def\psonedue{Monday, Feb. 11 at \duetime}
\def\pstwodue{Tuesday, Feb. 19 at \duetime}
\def\psthreedue{Monday, Feb. 25 at \duetime}
\def\psfourdue{Monday, Mar. 4 at \duetime}
\def\psfivedue{Monday, Mar. 18 at \duetime}
\def\pssixdue{Monday, Apr. 8 at \duetime}
\def\pssevendue{Wednesday, Apr. 17 at \duetime}
\def\pseightdue{Monday, Apr. 29 at \duetime}
\def\psninedue{Monday, May 6 at \duetime}
\def\pstendue{Not to be turned in}

\def\psetblurb{You are encouraged to discuss the problem sets with each other.
You must either list any collaborators or write "no collaborators" at the 
top of your paper.
You may \emph{not} read another student's paper for help with a problem.
You must write your solutions in your own words.}




%%%%%%%%%%%%%%%%%%%%%%%%%%%%%%%%%%%%%%%%%%

\newcommand{\mynull}[1]{}
\newcommand{\myquestA}[1]{#1}
\newcommand{\mysolA}[1]{#1}
% if hiding then print #1 else print #2
\newcommand{\myhidehide}[2]{#1}
\newcommand{\myhideshow}[2]{#2}

\def\myquest{\myquestA}
\def\mysol{\mysolA}
\def\myhide{\myhideshow}


\def\mybull{$\bullet$}
\def\W{\Omega}
\def\w{\omega}
\def\mycap{\,\cap\,}
\def\mycup{\,\cup\,}
\def\endtopic{\vspace*{\stretch{1}} \emph{End of class \classnum{} notes}}

%%%%%%%%%%%%%%%%%%%%%%%%%%%%%%%%%%
\def\classnum{2}
\def\topic{Probability: Terminology and Examples}

\def\parttitle{Class \classnum{}\\ \topic\\18.05, \whichterm }
\def\mypagehead{18.05 class \classnum, \topic, \whichterm }
\def\mytitle{\parttitle}

%%%%%%%%%%%%%%%%%%%%%%%%%%%%%%%%%%%
\pagestyle{headings}
\markboth{\mypagehead}{\mypagehead}

\begin{document}
\thispagestyle{plain}

\begin{center}
  \Large\bfseries \mytitle
\end{center}

%%%%%%%%%%%%%%%%%%%%%%%%%%%%%%%%%%%

\section{Learning Goals}
\begin{enumerate}[1.]
\item Know the definitions of sample space, event and probability function.
\item Be able to organize a scenario with randomness into an experiment and sample space.
\item  Be able to make basic computations using a probability function.
\end{enumerate}

\section{Terminology}

\subsection{Probability cast list}
\begin{enumerate}[\mybull]
\item Experiment: a repeatable procedure with defined outcomes.
\item Sample space: the set of all possible outcomes. We usually denote
the sample space by $\W$, sometimes by $S$.
\item Event: a subset of the sample space.
\item Probability function: a function giving the probability for each outcome.
\item Probability density: (We'll get to this later in the course.)
\item Random variable: a random numerical outcome (We'll get to this later in the course.)
\end{enumerate}

\subsection{Simple examples}
\numexamp Toss a fair  coin.\\
Experiment: toss the coin, report if it lands heads or tails.\\
Sample space: $\W = \{H,\, T\}$.\\
Probability function: $P(H) = .5$, \, $P(T) = .5$.

\medskip

\numexamp Toss a fair  coin 3 times.\\
Experiment: toss the coin 3 times, list the results.\\
Sample space: $\W = \{HHH, \, HHT, \, HTH, \, HTT, \, THH, \, THT, \, TTH, \, TTT\}$.\\
Probability function: Each outcome is equally likely with probability 1/8.

For small sample spaces we can put the set of outcomes and probabilities into
a table.

\begin{tabular}{l|cccccccc}
Outcomes    & HHH & HHT & HTH & HTT & THH & THT & TTH & TTT\\
\hline
Probability & 1/8 & 1/8 & 1/8 & 1/8 & 1/8 & 1/8 & 1/8 & 1/8
\end{tabular}
\medskip

\numexamp Measure the mass of a proton\\
Experiment: follow some procedure to measure the mass and report the result.\\
Sample space: $\W = [0,\infty)$, i.e. in principle we can get any positive value.\\
Probability function: sinces there is a continuum of possible outcomes there
is no probability function. Instead we need to use a 
\emph{probability density}, which we will learn about
later in the course.

\medskip

\numexamp (An infinite sample space)\\
Experiment: count the number of taxis that pass 77 Mass. Ave
during this class.\\
Sample space: $\W = $ \{0, 1, 2, 3, 4, \ldots\}.\\
This is often modeled with the following probability function (known as the
Poisson distribution):\\
\[\ds{P(k) = \e{-\lambda}\frac{\lambda^k}{k!}},\]
where $\lambda$ is the average number of taxis. 
We can put this in a table:

\begin{tabular}{l|cccccccc}
Outcome    & 0 & 1 & 2 & 3 & \ldots & k & \ldots\\[.6ex]
\hline\\[-.9ex]
Probability & $\e{-\lambda}$ & $\e{-\lambda}\,\lambda$ 
& $\e{-\lambda}\,\lambda^2/2$ & $\e{-\lambda}\,\lambda^3/3!$
& \ldots & $\e{-\lambda}\,\lambda^k/k!$ & \ldots
\end{tabular}

Question: Accepting that this is a valid probability function, what 
is $\ds{\sum_{k=0}^\infty \e{-\lambda}\frac{\lambda^k}{k!}}$ \,?\\
\ans This is the total probability of all possible outcomes, so the sum equals 1. (Note, this follows from the Taylor series \,
$\ds{\e{\lambda} = \sum_{n=0}^\infty \frac{\lambda^n}{n!}}$.)


\medskip

In a given setup there can be more than one possible sample space.

\numexamp (Choice of sample space)\\
Suppose you roll one die. Then the sample space and probability function are

\mcent{\begin{tabular}{l|cccccc}
Outcome & 1 & 2 & 3 & 4 &5 & 6\\
\hline
Probability: & 1/6 & 1/6 & 1/6 & 1/6 & 1/6 & 1/6
\end{tabular}}

Now, suppose you roll two dice. What should be the sample space? Here are two
options.\\
1. Record the pair of numbers showing on the dice (first die, second die). \\
2. Record the sum of the numbers on two dice. In this case there are
11 outcomes $\{2,3,4,5,6,7,8,9,10,11,12\}$. These outcomes are not equally
likely.

As above, we can put this information in tables. For the first case, the 
sample space is the product of the sample spaces for each die
\[ \{(1,1),\, (2,1),\, (3,1), \ldots (6,6)\}\]
Each of the 36 (why 36?) outcomes are equally likely. 
For the probability function we will
make a two dimensional table with the rows representing the number on the
first die, the columns the number on the second die and the entries the
probability.

\mcent{
\begin{tabular}{l|c|c|c|c|c|c|c|}
\multicolumn{7}{c}{\hs{14} Die 2}\\
\cline{2-8}
&  &  1  &  2  &  3  &  4  &  5  & 6\\
\cline{2-8}
&1 & 1/36 & 1/36 & 1/36 & 1/36 & 1/36 & 1/36\\
\cline{2-8}
&2 & 1/36 & 1/36 & 1/36 & 1/36 & 1/36 & 1/36\\
\cline{2-8}
Die 1&3 & 1/36 & 1/36 & 1/36 & 1/36 & 1/36 & 1/36\\
\cline{2-8}
&4 & 1/36 & 1/36 & 1/36 & 1/36 & 1/36 & 1/36\\
\cline{2-8}
&5 & 1/36 & 1/36 & 1/36 & 1/36 & 1/36 & 1/36\\
\cline{2-8}
&6 & 1/36 & 1/36 & 1/36 & 1/36 & 1/36 & 1/36\\
\cline{2-8}
\end{tabular}}

{\mcent{\hs5Two dice in a two dimensional table}}\\

For the second case we can present outcomes and probabilities in our usual table.\\

\mcent{
\begin{tabular}{l|ccccccccccc}
outcomes  &  2  &  3  &  4  &  5  &  6  & 7 & 8 & 9 & 10 & 11 & 12\\
\hline
probability & 1/36 & 2/36 & 3/36 & 4/36 & 5/36 & 6/36 & 5/36 & 4/36 
& 3/36 & 2/36 & 1/36
\end{tabular}
}\nl5
\mcent{The sum of two dice}\\

We will see that the best choice of sample space depends on the context. For now, simply
note that given the outcome as a pair of numbers it is easy to
find the sum.


\subhead{Note} Listing the experiment, sample space and probability function
is a good way to start working systematically with probability. It can help 
you avoid some of the common pitfalls in the subject.

\medskip

\subhead{Events}

We defined an  \emph{event} as a collection of outcomes. In other words
an event is a subset of the sample space $\W$. This sounds
odd, but it actually corresponds to the common meaning of the word.

\numexamp Using the setup in example 2 we would describe an
event in words by saying something like, $E$ = `the event
you get exactly 2 heads'. Written as a subset this becomes
\[E = \{HHT, \, HTH, \, THH\}.\]
You should get comfortable moving between describing events
in words and as subsets of the sample space.

In this example the probability of $E$ is computed by
adding up the probabilities of all of the outcomes in
$E$. Since each outcome has probability 1/8, we have $P(E) = 3/8$.

\subsection{Definition of a discrete sample space}
\subhead{Definition} \, A \textbf{discrete sample space} is one that is 
listable, it can be either finite or infinite.

\examples \{H, T\}, \, \{1, 2, 3\}, \, \{1, 2, 3, 4, \ldots\}, \,
\{2, 3, 5, 7, 11, 13, 17, \ldots\} are all discrete sets. The first two
are finite and the second two are infinite.

\example The interval $0 \le x \le 1$ 
is \emph{not} discrete, rather it is \emph{continuous.} We will deal with 
continuous sample spaces in a few days.

\subsection{The probability function}

So far we've been using a casual definition of the probability function.

\subhead{Careful definition of the probability function}\\
For a discrete sample space a \emph{probability function} assigns to each
outcome $\w$ a number $P(\w)$ called the probability. It must satisfy:\\
1. $0 \le P(\w) \le 1$ \, (the probability is between 0 and 1).\\
2. The sum of the probabilities of all possible outcomes is 1 (something must

In symbols rule 2 says: if $\W = \{\w_1, \w_2, \ldots, \w_n\}$ then
$P(\w_1) + P(\w_2) + \ldots + P(\w_n) = 1.$  Using summation notation:
$\ds{\sum_{j=1}^n P(\w_j) = 1}$.

The probability of an event $E$ is the sum of the probabilities of 
all the outcomes in $E$.

\problem Check rules 1 and 2 on examples 1 and 2 above.

\medskip


\numexamp (A classic example)\\
Suppose we have a coin with probability $p$ of heads.\\
Experiment: Toss the coin until the first heads. Report the number of tosses.\\
Sample space: $\W = $ \{1, 2, 3, \ldots\}.\\
Probability function: $P(n) = (1-p)^{n-1}p$.

Challenge 1: show the sum of all the probabilities equals 1 (hint: geometric series).\\
Challenge 2: justify the formula for $P(n)$ (we will do this soon).

\subhead{Stopping problems}
As usual with our
toy examples, the previous example is an uncluttered version of a general
class of problems called \textbf{stopping rule problems}. A
stopping rule is a rule that tells you when to end a certain
process.  In the example the process was flipping a coin and 
we stopped after the first heads.  A
possibly more practical sounding example might be a rule for
ending a series of medical treatments.  One could ask about
the probability of stopping within a certain number of
treatments or the average number of treatments you should
expect.


\section{Some rules of probability}
For events $A$, $L$, $R$
\begin{enumerate}[Rule 1.]
\item $P(A^c)  = 1 - P(A)$.
\item If $L$ and $R$ are disjoint then $P(L \cup R)  = P(L) + P(R)$.
\item If $L$ and $R$ are not disjoint, we have the \textbf{inclusion-exclusion principle}: $$P(L\mycup R)  =  P(L) + P(R) - P(L\mycap R)$$
\end{enumerate}

We visualize these rules using Venn diagrams.

\mcent{
\includegraphics{\imgdir/figc2-1.pdf}
\includegraphics{\imgdir/figc2-2.pdf}
\includegraphics{\imgdir/figc2-3.pdf}
}

We can also justify them logically.

Rule 1: 
$A$ and $A^c$ split $\W$ into
two non-overlapping regions. Since the total probability  $P(\W) = 1$ 
this rule says that the probabiity of $A$ and the probability of 'not $A$'
are complementary, i.e. sum to 1.

Rule 2: 
$L$ and $R$ split $L\cup R$ into
two non-overlapping regions. So the probability of $L\cup R$ is
is split between $P(L)$ and $P(R)$

Rule 3: 
In the sum $P(L) + P(R)$ the overlap $P(L\cap R)$ gets counted twice.
So $P(L) + P(R) - P(L\cap R)$ counts everything in the union exactly once.

In the following example problems we have an experiment that produces
a random integer between 1 and 20. The probabilities are not 
necessarily uniform, i.e., the same for each outcome.

\numexamp  If the probability of an even number is .6 what is the probability of an odd number.

\ans Since being odd is complementary to being even, the probability of being odd is 1-.6 = .4.

Let's redo this example a bit more formally, so you see how it's done.
First, so we can refer to it, let's name the random integer $X$. 
Let's also name the event `$X$ is even' as $A$. Then the event
`$X$ is odd' is $A^c$.
We are given that  $P(A) = .6$. Therefore $P(A^c) = 1-.6 = \boxed{.4}$.

\numexamp Consider the 2 events, $A$: `$X$ is a multiple of 2'; $B$: `$X$ is odd and less than 10'. Suppose $P(A)=.6$ and $P(B)=.25$.\\
(i) What is $A\mycap B$?\\
(ii) What is the probability of $A\mycup B$? 

\ans (i) Since all numbers in $A$ are even and all numbers in $B$ are odd, these events are disjoint. That is, $\boxed{A\mycap B = \emptyset.}$

(ii) Since $A$ and $B$ are disjoint $\boxed{P(A\mycup B) = P(A) + P(B) = .85.}$

\medskip

\numexamp Let $A$, $B$ and $C$ be the events $X$ is a multiple of 2, 3 and 6
respectively. If $P(A)=.6$, $P(B)=.3$ and $P(C)=.2$ what is $P(A \text{ or } B)$?

\ans Note two things. First we used the word `or' which means union:
 `$A$ or $B$' = $A\mycup B$. 
Second, an integer is divisible by 6 if and only if it is divisible by both 2 and 3.  This translates into $C = A\mycap B$. So the inclusion-exclusion principle says 
\[ P(A\mycup B) = P(A) +P(B) - P(A\mycap B) = .6 + .3 - .2 = \boxed{.7}.\]
\endtopic
\end{document}
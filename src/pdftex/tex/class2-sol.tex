



% Solutions to in class problems

\numexamp 
In poker, the number of one-pair hands is
$$\binom{13}{1}\binom{4}{2}\binom{12}{3}\binom{4}{1}^3.$$
Carefully justify this count using the rule or product.  Use this to express the probability of a one-pair hand. 

\ans By the rule of product, the answer is the product of the number of
ways to do each of the following steps.\\
First choose 1 rank of of 13 to be the pair: $\binom{13}{1}$\\
Then choose 2 out of the 4 cards from that rank: $\binom{4}{2}$\\
Then choose 3 different ranks from  the 12 remaining ones: $\binom{12}{3}$
For each of those ranks choose 1 out of 4 cards. $\binom{4}{1}
\cdot\binom{4}{1}\cdot\binom{4}{1}$.


\numexamp 
What is the probability of getting a one-pair hand in poker?

\ans The total number of hands is $\binom{52}{5}$.  
Using Matlab we compute this as 2598960. Also using Matlab we
find the number of one-pair hands computed above is 1098240
The probability of 1-pair is therefore the ratio: 1098240/2598960
= .42


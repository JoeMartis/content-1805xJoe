%%%%%%%%%%%%%%%%%%%%%%%%%%%%%%%%%%%%%%%%%%
%% Version 5
%    Jan 3, 2013
%       Added \subhead, \problem, \numexamp
%    Feb 27, 2012
%       Added \xbar, \Var, \Cov  macros
%    Feb 11, 2012
%       Reorganized to use with header.tex and new mkpdf.sh
%       added some macros
%    Feb 5, 2012
%       Copied from 18.03
%       Added myhide code.
%       Removed topics codes
%%%%%%%%%%%%%%%%%%%%%%%%%%%%%%%%%%%%%%%%%%

%%  Macros
\def\partHeading#1#2{\mcent{\textrm{\large \bf Part #1 \hspace{2.3ex}#2}}}


%%  Simple macros
\def\mycomment#1{{\color{red}#1}}
\def\mcent#1{\hspace*{\stretch{1}}#1\hspace*{\stretch{1}}{ }}
\def\hs#1{\hspace*{#1 ex}}
\def\nl#1{\\[.#1ex]}
\def\ds{\displaystyle}
\def\cont{\vspace*{\stretch{1}}\emph{(continued)} \newpage}
\def\reading{\textbf{Reading. }}
\def\examples{\textbf{Examples. }}
\def\example{\textbf{Example. }}
\def\examp#1{\textbf{Example #1. }}
\def\problem{\textbf{Problem. }}
\def\problems{\textbf{Problems. }}
\def\subhead#1{\textbf{#1.}}
\def\indetop#1#2#3{\stackrel{\; \scriptstyle{{#1}{#3}{#2}}}{=} \;}
\def\indet#1#2{\stackrel{\scriptstyle{\; \frac{#1}{#2}}}{=} \;}
\def\indetzero{\indet{0}{0}}
\def\indetinf{\indetop{\infty}{\infty}{/}}
\def\qedbox{\rule{1ex}{1ex}}

\def\mybull{$\bullet$}
\def\W{\Omega}
\def\w{\omega}
\def\mycap{\,\cap\,}
\def\mycup{\,\cup\,}
\def\Norm{\text{N}}

\def\myref#1{(\ref*{#1})}

\def\mypart#1#2{\frac{\partial #1}{\partial #2}}
%      fully specified partial
\def\fp#1#2#3{(\partial #1/\partial #2)_{#3}}                %without \frac
\def\ffp#1#2#3{\left.\frac{\partial #1}{\partial #2}\right|_{#3}}  %with \frac
\def\mpp#1#2#3{\left(\mypart{#1}{#2}\right)_{#3}} 
\def\myderiv#1#2{\frac{d#1}{d#2}}
\def\mynderiv#1#2#3{\frac{d^{#3}#1}{d#2^{#3}}}
\def\mygrad{\boldsymbol{\nabla}}         %gradient
\def\gf#1#2{\left.\mygrad #1\right|_{#2}}
\def\endandindent{\\ \hspace*{20pt}}
\def\e#1{\mathrm{e}^{#1}}
\def\myIm{\textup{Im}}
\def\myRe{\textup{Re}}
\def\conj#1{\overline{#1}}
\def\tran{^\mathrm{T}}
\def\ans{{\bf \underline{answer:}}{ }}
\def\th{$^{\mathrm{th}}$}
\def\myhead#1{\noindent \textbf{#1}}
\def\tbf#1{\textbf{#1}}
\def\mysep{: }
\def\lap{{\mathcal{L}}}
\def\ilap{\lap^{-1}}
\def\myimply{\; \Rightarrow \;}
\def\myequiv{\; \Leftrightarrow \;}
\def\ft{\hat}
\def\xbar{\overline{X}}
\def\Var{\textup{Var}}
\def\Cov{\textup{Cov}}
\def\bypartshelp#1#2#3#4#5#6#7{\framebox{$\begin{array}[#7]{lll} 
#5=#1 & d#6=#2\\
d#5=#3 & #6=#4 \end{array}$}}
\def\byparts#1#2#3#4{\bypartshelp{#1}{#2}{#3}{#4}{u}{v}{l}}
\def\bypartst#1#2#3#4{\bypartshelp{#1}{#2}{#3}{#4}{u}{v}{t}}

%matlab 
\newcommand{\mlcmd}[2]{\textgreater{} \texttt{#1}\\ \texttt{#2}\\}
\newcommand{\mlcmdna}[1]{\textgreater{} \texttt{#1}\\}
\newcommand{\mlans}[2]{#1 =\\[.5ex]\hs3\parbox{5in}{#2}}
\def\mlcomm#1{{\color{blue}\% #1}\\}
\def\mlinstr#1{\mlcomm{#1}}
\def\mlcar{\^{ }}

%counters
\newcounter{examplectr}
\setcounter{examplectr}{1}
\renewcommand{\theexamplectr}{\arabic{examplectr}}
\newcommand{\numexamp}{\textbf{Example \arabic{examplectr}.{ }}\addtocounter{examplectr}{1}}

%matrices
\def\defleftbrace{(}
\def\defrightbrace{)}
\def\twobytwohelp#1#2#3#4#5#6#7{\left#6 \begin{array}{#1} #2 & #3 \\ #4 & #5 \end{array} \right#7}
\def\twobytwo#1#2#3#4{\twobytwohelp{rr}{#1}{#2}{#3}{#4}{\defleftbrace}{\defrightbrace}}
\def\twobytwoc#1#2#3#4{\twobytwohelp{cc}{#1}{#2}{#3}{#4}{\defleftbrace}{\defrightbrace}}
\def\twobytwodet#1#2#3#4{\twobytwohelp{rr}{#1}{#2}{#3}{#4}{|}{|}}
\def\twobytwodetc#1#2#3#4{\twobytwohelp{cc}{#1}{#2}{#3}{#4}{|}{|}}
\def\twobyonehelp#1#2#3{\left\defleftbrace \begin{array}{#1} #2\\ #3  \end{array} \right\defrightbrace}
\def\twobyone#1#2{\twobyonehelp{r}{#1}{#2}}
\def\twobyonec#1#2{\twobyonehelp{c}{#1}{#2}}
\def\threebythree#1#2#3#4#5#6#7#8#9{\left\defleftbrace \begin{array}{rrr} #1&#2&#3\\ #4&#5&#6\\ #7&#8&#9 \end{array} \right\defrightbrace}
\def\threebythreec#1#2#3#4#5#6#7#8#9{\left\defleftbrace \begin{array}{ccc} #1&#2&#3\\ #4&#5&#6\\ #7&#8&#9 \end{array} \right\defrightbrace}
\def\threebythreedet#1#2#3#4#5#6#7#8#9{\left| \begin{array}{rrr} #1&#2&#3\\ #4&#5&#6\\ #7&#8&#9 \end{array} \right|}
\def\threebythreedetc#1#2#3#4#5#6#7#8#9{\left| \begin{array}{ccc} #1&#2&#3\\ #4&#5&#6\\ #7&#8&#9 \end{array} \right|}
\def\threebyone#1#2#3{\left\defleftbrace \begin{array}{r} #1\\ #2\\ #3 \end{array} \right\defrightbrace}
\def\threebyonec#1#2#3{\left\defleftbrace \begin{array}{c} #1\\ #2\\ #3 \end{array} \right\defrightbrace}
\def\threebytwo#1#2#3#4#5#6{\left\defleftbrace \begin{array}{rr} #1&#2\\ #3&#4\\ #5&#6 \end{array} \right\defrightbrace}
\def\threebytwoc#1#2#3#4#5#6{\left\defleftbrace \begin{array}{cc} #1&#2\\ #3&#4\\ #5&#6 \end{array} \right\defrightbrace}

%Vectors and line segments
\def\vb#1{\mathbf{#1}}  %bold
\def\va#1{\overrightarrow{#1}}  %arraow
\def\vba#1{\overrightarrow{\mathbf{#1}}}  %bold/arraow
\def\vl#1{\overline{#1}} %overline
\def\vbl#1{\overline{\mathbf{#1}}} %bold/overline
\def\vu#1{\widehat{#1}} %unit vector
\def\vbu#1{\widehat{\mathbf{#1}}} %bold/unit vector
\def\un#1{\frac{#1}{\left|#1\right|}}  %normalize vector
\def\vbi{\vbu{i}}
\def\vbj{\vbu{j}}
\def\vbk{\vbu{k}}
\def\vc#1{\langle #1 \rangle}
\def\vcb#1{\left\langle #1 \right\rangle}

\newcommand{\st}[1]{\ensuremath{^{\scriptstyle \textrm{#1}}}}
\newcommand{\undertext}[1]{\ensuremath{\underline{\textrm{#1}}}}
\newcommand{\fracc}{\displaystyle\frac}
\newcommand{\summ}{\displaystyle\sum}

%%%%%%%%%%%%%%%%%%%%%%%%%%%%
%% tikz
\usetikz{\tikzset{
addarrow/.style={postaction={decorate},
        decoration={markings,mark=at position #1 with {\arrow{>}}}}
}}

%%%%%%%%%%%%%%%%%%%%%%%%%%%
%% Sample Complex formatting macros
%\makeatletter
%\renewcommand\section{\goodbreak\@startsection {section}{1}{\z@}%
%%%                                   {-3.5ex \@plus -1ex \@minus -.2ex}%
%                                   {-3.5ex \@plus -1ex \@minus -.2ex}%
%                                   {2.3ex \@plus.2ex}%
%                                   {\normalfont\large\bfseries%
%                                     \centering\sectionname\ }}

%\renewcommand\subsection{\goodbreak\@startsection{subsection}{2}{\z@}%
%%                                     {-3.25ex\@plus -1ex \@minus -.2ex}%
 %                                    {-2ex\@plus -1ex \@minus -.2ex}%
%%                                     {1.5ex \@plus .2ex}%
%                                     {1.25ex \@plus .2ex}%
%                                     {\normalfont\large\bfseries}}
%\makeatother

%\renewcommand{\thesection}{\Roman{section}}
%\newcommand\sectionname{Part}
%\newcommand{\alphalist}{% changes enumerate 1st level to (a)...(z)
%  \renewcommand{\theenumi}{\alph{enumi}}%
%  \renewcommand{\labelenumi}{\theenumi)}%
%}
%\alphalist
%%%%%%%%%%%%%%%%%%%%%%%%%%%%%%%%%%%%%%%%%%

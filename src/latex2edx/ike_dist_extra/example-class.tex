%
% File:   example-template-class.tex
% Date:   17-Jul-12
% Author: I. Chuang <ichuang@mit.edu>
%
% Updated 23-Jan-13
%
% Example latex source file for an edX course.
% This file can be compiled using
%
%   python latex2edx -prefix ATEST example-template-class.tex
%
% to generate all the content files for an example edX course named
% "1.00x".  The files are saved in a subdirectory named "1.00x", and
% include, for example
%
%     1.00x/course.xml               - main table of contents for course
%     1.00x/problem/Problem_1.xml   - problem 1 (symbolic response example)
%     1.00x/problem/Problem_2.xml   - problem 2 (option response example)

\documentclass[12pt]{article}

\usepackage{edXpsl}	% edX

%%%%%%%%%%%%%%%%%%%%%%%%%%%%%%%%%%%%%%%%%%%%%%%%%%%%%%%%%%%%%%%%%%%%%%%%%%%%%
\begin{document}

% Arguments to edXcourse are: course number, course semester ``url-name''
% the ``url-name'' should match the directory name in the ``policies''
% directory.  Policies are not currently set by parameters in this file.

\begin{edXcourse}{6.341x}{2013_Fall}

% A chapter is like a week of a course, or a conceptual unit of a course.
% Chapters are listed as headings in the left-hand navigation bar.

\begin{edXchapter}{Unit 1}

% A section is like a day of a week, or a conceptual subunit of a chapter.
% Sections are subheadings in the left-hand navigation bar
% A section may contain multiple pieces of content, eg video, problem, html
% Each piece of content within a section is listed as one element
% in the sequential navigation bar going from left to right, across the top
% of the problem window.

\begin{edXsection}{Introduction}

%%%%%%%%%%%%%%%%%%%%%%%%%%%%%%%%%%%%%%%%
% edXtext = html page
% The argument for edX text specifies the ``display name'' for the page

\begin{edXtext}{Introductory Text}

\section{Course Highlights}

This course features a complete set of lecture notes and assignments
which tie directly into the required textbook: Oppenheim and Schafer
with Buck, Discrete-Time Signal Processing, 2nd ed, Upper Saddle
River, NJ: Prentice-Hall, 1999, ISBN: 0137549202.

\section{Course Description}

This class addresses the representation, analysis, and design of
discrete time signals and systems. The major concepts covered include:
Discrete-time processing of continuous-time signals; decimation,
interpolation, and sampling rate conversion; flowgraph structures for
DT systems; time-and frequency-domain design techniques for recursive
(IIR) and non-recursive (FIR) filters; linear prediction; discrete
Fourier transform, FFT algorithm; short-time Fourier analysis and
filter banks; multirate techniques; Hilbert transforms; Cepstral
analysis and various applications.

\end{edXtext}
%%%%%%%%%%%%%%%%%%%%%%%%%%%%%%%%%%%%%%%%

%%%%%%%%%%%%%%%%%%%%%%%%%%%%%%%%%%%%%%%%
% An edXproblem = problem
% The arguments are {display name}{keyword values}
%
% keyword values which can be set are any metadata settings, like npoints (number of points)
% if ``keyword values'' is just a number, that is taken to be the problem's number of points.
%
% The key macro is ``edXabox'', which specifies an ``answer box'' for a problem.
% Each problem may have one or more answer boxes.
%
\begin{edXproblem}{Example symbolic response}{npoints=40}

This is a sample problem, which is worth 40 points.  What is the number made famous by
Hitchhiker's guide to the galaxy?

\edXabox{type='symbolic' size='90' expect='42' }

\end{edXproblem}

%%%%%%%%%%%%%%%%%%%%%%%%%%%%%%%%%%%%%%%%

\begin{edXproblem}{Example option response}{10}

This is a sample problem, which is worth 10 points.

Give the correct python {\tt type} for the following expressions.  Select {\tt noneType} if the expression is illegal.

\begin{itemize}
\item {\tt 3}   \edXabox{expect="int" options="noneType","int","float"}
\item {\tt 5.2} \edXabox{expect="float" options="noneType","int","float"}
\item {\tt 3/2} \edXabox{expect="int" options="noneType","int","float"}
\item {\tt 1+[]} \edXabox{expect="noneType" options="noneType","int","float"}
\end{itemize}

\end{edXproblem}

%%%%%%%%%%%%%%%%%%%%%%%%%%%%%%%%%%%%%%%%

\begin{edXproblem}{Example string response}{10}

What state is Detroit in?

\edXabox{expect="Michigan" type="string"}

\end{edXproblem}

%%%%%%%%%%%%%%%%%%%%%%%%%%%%%%%%%%%%%%%%
% A ``vertical'' is a set of multiple problems stacked above each other, eg
% multiple problems for one day.

\begin{edXvertical}{A sample vertical - multiple problems in one day}

%%%%%%%%%%%%%%%%%%%%%%%%%%%%%%%%%%%%%%%%

\begin{edXproblem}{Example numerical response}{10}

What is the numerical value of $pi$?

\edXabox{expect="3.14159" type="numerical" tolerance='0.01' }

\end{edXproblem}

%%%%%%%%%%%%%%%%%%%%%%%%%%%%%%%%%%%%%%%%

\begin{edXproblem}{Example of a custom response}{10}

This problem demonstrates the use of a custom python script used for
checking the answer.

\begin{edXscript}
def sumtest(expect,ans):
    (a1,a2) = map(float,eval(ans))
    return (a1+a2)==10
\end{edXscript}

Enter a python list of two numbers which sum to 10:

\edXabox{expect="[1,9]" type="custom" cfn="sumtest"}

\end{edXproblem}

%%%%%%%%%%%%%%%%%%%%%%%%%%%%%%%%%%%%%%%%

\end{edXvertical}

\end{edXsection} % end of introduction

%%%%%%%%%%%%%%%%%%%%%%%%%%%%%%%%%%%%%%%%%%%%%%%%%%%%%%%%%%%%
\begin{edXsection}{Day 2}

%%%%%%%%%%%%%%%%%%%%%%%%%%%%%%%%%%%%%%%%

\begin{edXproblem}{Example inline textinput answer box}{40}

\section{Inline Answer Box Test}  

\begin{edXscript}

def dumtest(expect,ans):
    return True

\end{edXscript}

Clifford gates and measurements are not the only components of quantum
circuits. In fact, if they were, quantum computers not be more
powerful than classical ones! Here we consider some unitary and
non-unitary operations that take quantum states to other valid states.

\subsection{Rotations}

Recall the Bloch vector $[a,b,c]$ for a single qubit state $\rho$ is defined so that 
\begin{equation}
\rho = \frac12(I+aX+bY+cZ)
\end{equation}
where $I$ is the $2\times2$ identity matrix and $X,Y,$ and $Z$ are the
Pauli matrices. The purity of state $\tau$ for $n$-qubits is the trace
of $\tau^2$, a quantity which is always between $1/2^n$ and $1$, with
$1$ being pure and $1/2^n$ being completely mixed. What is the
condition on $a,b,$ and $c$ such that $\rho$ is a pure state? Enter
you answer as a single expression that will be 0 if (and only if)
$\rho$ is pure. For example, $(4-a+b+c^4)$ is syntactically correct
but $(4-a+b+c^4)=0$ is not.


\edXinline{$\rho$ is pure iff $0=$ } \edXabox{inline='1' type="custom" cfn='dumtest' expect="a^2 + b^2 + c^2 - 1"}

All single qubit unitary operations can be expressed, up to a global
phase, as a rotation about an axis in Bloch sphere space. If the axis
is $\hat n=[n_x,n_y,n_z]$ (a unit vector) and the rotation angle is
$\theta$, then the rotation may be written in terms of the Pauli
matrices as
\begin{equation}
U=e^{i\theta\left(n_x X+n_y Y+n_z Z\right)}.
\end{equation}

\end{edXproblem}


\end{edXsection}
\end{edXchapter}
%%%%%%%%%%%%%%%%%%%%%%%%%%%%%%%%%%%%%%%%%%%%%%%%%%%%%%%%%%%%
% note that an empty line must preceed the \begin{edXchapter} and the \begin{edXsection}

\begin{edXchapter}{Unit 2}

\begin{edXsection}{A new day}


%%%%%%%%%%%%%%%%%%%%%%%%%%%%%%%%%%%%%%%%

\edXvideo{display_name="Signals and Systems Part II" youtube= '1.0:jGk3w1b7UXQ'}

%%%%%%%%%%%%%%%%%%%%%%%%%%%%%%%%%%%%%%%%

\end{edXsection}
\end{edXchapter}
%%%%%%%%%%%%%%%%%%%%%%%%%%%%%%%%%%%%%%%%%%%%%%%%%%%%%%%%%%%%
% note that an empty line must preceed the \begin{edXchapter} and the \begin{edXsection}

\begin{edXchapter}{Unit 3}

\begin{edXsection}{Complex symbolic math checking}

%%%%%%%%%%%%%%%%%%%%%%%%%%%%%%%%%%%%%%%%

\begin{edXproblem}{Custom response with multiple answer boxes}{npoints=40}

\begin{edXscript}
def test_add(expect,ans):
  a1=float(ans[0])
  a2=float(ans[1])
  return (a1+a2)== float(expect)
\end{edXscript}

Enter two integers which sum to 20:

\edXabox{expect="20" answers="11","9" type="custom" prompts="Integer 1:","Integer 2:" cfn="test_add" inline='1'}

\edXsolution{An acceptable answer would be the two numbers 11 and 9.}

\end{edXproblem}

%%%%%%%%%%%%%%%%%%%%%%%%%%%%%%%%%%%%%%%%

\begin{edXproblem}{Complex symbolic response}{npoints=40}

\begin{edXscript}
from formula import *

def test_trace(expect, given):
    # ans should be a symmath matrix expression, like [[1,2],[2,3]]
    msg = ''
    try:
        xgiven = my_sympify(given, matrix=True)
        xt = xgiven.trace()
    except Exception, err:
        return {'ok': False, 'msg': 'Error %s<br/> in evaluating your expression "%s"' % (err, given)}
    return xt==1

\end{edXscript}

Enter a matrix with trace equal to $1$:

\edXabox{expect="1" type="custom" cfn="test_trace" math="1"}

\end{edXproblem}

%%%%%%%%%%%%%%%%%%%%%%%%%%%%%%%%%%%%%%%%

\begin{edXproblem}{Symbolic response with multiple input boxes}{npoints=40}

\begin{edXscript}
from formula import *
from sympy import *

def test_mprod(expect, given):
    msg = ''
    try:
        xgiven1 = my_sympify(given[0], matrix=True)
        xgiven2 = my_sympify(given[1], matrix=True)
        xp = xgiven1 * xgiven2
    except Exception, err:
        return {'ok': False, 'msg': 'Error %s<br/> in evaluating your expression "%s"' % (err, given)}
    return xp==eye(2)

\end{edXscript}

Enter two $2\times 2$ matrices whose product is the identity:

\edXabox{expect="20" answers=".","." type="custom" prompts="Matrix 1:","Matrix 2:" cfn="test_mprod" inline='1' math='1'}

\edXsolution{An acceptable answer would be the two matrices input as ${\tt [[0,1],[1,0]}$ and ${\tt [[0,1],[1,0]}$.}

\end{edXproblem}

%%%%%%%%%%%%%%%%%%%%%%%%%%%%%%%%%%%%%%%%

\end{edXsection}
\end{edXchapter}
\end{edXcourse}

%%%%%%%%%%%%%%%%%%%%%%%%%%%%%%%%%%%%%%%%%%%%%%%%%%%%%%%%%%%%%%%%%%%%%%%%%%%%%

\end{document}

\documentclass[12pt]{article}

\usepackage{edXpsl}	% edX

\usepackage[T1]{fontenc}
\usepackage{amssymb}
\usepackage{amsmath}
\usepackage{bm}
\usepackage{calc}
%\usepackage{enumerate}	% edX
%\usepackage{paralist}	% edX
\usepackage{cancel}
\usepackage{amsmath, amsthm, amsfonts, amssymb, color, mathrsfs,comment} % edX
%\usepackage{paralist, enumerate, amsmath, amsthm, amsfonts, amssymb, color, mathrsfs,comment}

\usepackage{lastpage}
%\usepackage{fancyhdr}
%\pagestyle{fancy}
%\fancyhf{} % clear all header and footer fields
%\fancyfoot[R]{\footnotesize Page \thepage\ of \pageref{LastPage}} %OBS: Fill in exam length manually!
%\renewcommand{\headrulewidth}{0pt}
%\renewcommand{\footrulewidth}{0pt}


\setlength{\textheight}{9in-\footskip}
\setlength{\textwidth}{6.5in}
\setlength{\evensidemargin}{0.5in}
\setlength{\oddsidemargin}{0.5in}
\setlength{\topmargin}{0pt}
\setlength{\headheight}{0pt}
\setlength{\headsep}{0pt}
\renewcommand{\baselinestretch}{1.5}
\setlength{\parindent}{0pt}

\newcounter{probctr}
\setcounter{probctr}{1}
\renewcommand{\theprobctr}{\arabic{probctr}}
%\newcommand{\problem}{\noindent\textbf{\arabic{probctr}.{ } }\addtocounter{probctr}{1}}
% latex2edx.py doesn't handle this command. We'll have to number problems by hand.
\newcommand{\problem}[1]{\textbf{#1.}}

\newcommand{\problemOLD}[2]{%		% edX
  \addtocounter{probctr}{1}
   \begin{list}{}{%
          \labelsep=1.1ex
          \settowidth{\labelwidth}{\textbf{99~(99 pts.)}}
          \addtolength{\labelwidth}{1em} 
          \leftmargin=\labelwidth 
          \addtolength{\leftmargin}{\labelsep} 
          \renewcommand{\makelabel}[1]{ ##1 }}
          \item[\textbf{\theprobctr ~ (#1 pts.)} \hfill]
       #2 
%       \newpage
   \end{list}
}


% \newenvironment{sol}{\begin{proof}[Solution]}{\end{proof}}	% edX
\newenvironment{sol}{\begin{edXsolution}}{\end{edXsolution}}	% edX

%---------------------------------------------------%
% Simple macros

% For matlab
\def\gt{\textgreater}
\def\lt{\textless}
\def\textcaret{\textasciicircum}
\def\textbl{\textbackslash}
\def\bracel{\textbraceleft}
\def\bracer{\textbrageright}
\def\textunder{\textunderscore}
%\def\textstar{\textasteriskcentered} don't use: latex2edx doesn't parse it just use plain *

\newcommand{\mlprompt}{\textgreater{}}
\newcommand{\mlcaret}{\textasciicircum{}}
\newcommand{\mlcmd}[2]{\textgreater{} \texttt{#1}\\ \texttt{#2}\\}
\newcommand{\mlcmdna}[1]{\textgreater{} \texttt{#1}\\}
\newcommand{\mlans}[2]{#1 =\\[.5ex]\hs3\parbox{5in}{#2}}
\def\mlcomm#1{\mycomment{\% #1}\\}


%%%%%%%%%%%%%%%%%%%%%
%Class 2 reading questions
%%%%%%%%%%%%%%%%%%%%

\newcommand{\RR}{{\mathbb R}}
\newcommand{\bs}[1]{\hbox{\boldmath$#1$}}
\DeclareRobustCommand{\transp}{^{\rm T}}
\newcommand{\lsub}[1]{{\hbox{\footnotesize$#1$}}}
\newcommand{\lexp}[1]{{\hbox{\footnotesize$#1$}}}
\newcommand{\blankpagemark}{\centerline{\sl This page intentionally blank.}}
\DeclareRobustCommand{\prm}{^{\hbox{\footnotesize$\,\prime$}}}
\DeclareRobustCommand{\pprm}{^{\hbox{\footnotesize$\,\prime\prime$}}}
\newenvironment{m}[1]{\left[\begin{array}{@{\,}#1@{\,}}}{\end{array}\right]}
%\newcommand{\text}{\mbox}
\newcommand{\mat}[1]{\begin{bmatrix} #1 \end{bmatrix}}

%Hacks so pdflatex and latex2edx will run
\def\bs{\textbackslash{}}

%------------------------------------------------------------------------<

\begin{document}
\thispagestyle{empty}

%%%%%%%%%%%%%%%%%%%%%%%%%%%%%%%%%%%%%%%%%%%%%%%%%%%%%%%%%%%%%%%%%%%%%%%%%%%%%

\begin{edXproblem}{c2-rq1}{weight=5  showanswer='attempted' display_name="Class 2 Reading Questions" }

\problem1
Which of the following represents a valid probability table 

\begin{tabular}{ll|lllllll}
i) & outcomes & 1 &2 &3 &4 &5 \\
\cline{2-7}
&probability &1/5 &1/5 &1/5 &1/5 &1/5
\end{tabular}

\bigskip

\begin{tabular}{ll|lllllll}
ii) & outcomes & 1 &2 &3 &4 &5 \\
\cline{2-7}
&probability &1/2 &1/5 &1/10 &1/10 &1/10
\end{tabular}

\edXabox{type="multichoice" expect="i and ii" options="i","ii","i and ii","not enough information" }

\problem2
A sample space has exactly two outcomes. The first has probability $p$
and the second has probability $3p$. What is the value of $p$?

\edXabox{expect=".25" type="numerical" tolerance='0.0001'}

\problem3
Let $A$ and $B$ be two events with $P(A) = 0.4$, $P(B) = 0.5$ and 
$P(A\cap B) = 0.3$. What is $P(A \cup B)$?
(Hint: visualize with Venn diagrams.)

\edXabox{type="multichoice" expect="0.6" options="0.5","0.6","0.7","0.8","0.9" }

\problem4
Our experiment consists of tossing a coin 3 times. Take the sample space
to be all sequences of 3 heads or tails.
\[ \Omega = \{HHH,\,HHT,\,HTH,\,HTT,\,THH,\,THT,\,TTH,\,TTT\}
\]
For each of the following give the size of the event.

i) The event 'there are more heads than tails'.

\edXabox{type="multichoice" expect="4" options="0","1","2","3","4","5","6","7","8"}

ii) The event 'the first heads occurs after the first tails.'

\edXabox{type="multichoice" expect="3" options="0","1","2","3","4","5","6","7","8"}


\end{edXproblem}

%\problem4 Now we'll do our first random simulation. Don't worry if you
%don't fully understand the commands. We will explain them in 
%the first tutorial. 
%
%In order to simulate counting the number of heads in 1000 tosses of  a fair coin give the following sequence of commands (hit return after each one).
%
%(In the first command the semicolon at the end 
%will keep a long list of numbers from being
%written to the screen. If you leave it off, you'll need to use the space bar to
%page through all the numbers.)\\
%\mlcmdna{x = binornd(1,.5, 1000,1);}
%\mlcmdna{total = sum(x)}
%
%You can use the up arrow to go back and run the commands again. You should get
%a different total.
%
%Give the value of the total number of heads for one of your simulations.
%\edXabox{expect="500" type="numerical" tolerance='60'}

\end{document}





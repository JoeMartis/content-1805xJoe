\documentclass[\ptsz]{article}

%%%%%%%%%%%%%%%%%%%%%%%%%%%%%%%%%%%%%%%%%%
%%  Packages
%\def\usetikz#1{}
\def\usetikz#1{#1}
\usepackage{amssymb,amsmath}
\usepackage{graphicx}  %needed for the includegraphics command
\usepackage{verbatim} % needed for comment environment
\usepackage{ifthen} % needed for ifthenelse
\usepackage{hyperref} %for comments
\usepackage{xcolor} %for comments
\usepackage{enumerate}% for lettering equations
\usetikz{\usepackage{tikz}
\usepgflibrary{arrows}
\usetikzlibrary{calc,through,backgrounds} %****no spaces after commas***
\usetikzlibrary{decorations.pathmorphing}
\usetikzlibrary{decorations.markings}
\usetikzlibrary{positioning} %for below=of xxx syntax in nodes
\usetikzlibrary{backgrounds,fit} %for background layers example
}
\usepackage[all,arc]{xy} %load this one last
%%%%%%%%%%%%%%%%%%%%%%%%%%%%%%%%%%%%%%%%%%

%%%%%%%%%%%%%%%%%%%%%%%%%%%%%%%%%%%%%%%%%%
%%  Settings

\textwidth=6in
\textheight=8.9in
\topmargin=-0.5in
\evensidemargin=0in
\oddsidemargin=0in
\parindent=0pt
\parskip=1ex

\unitlength =  10.0 pt 
%%%%%%%%%%%%%%%%%%%%%%%%%%%%%%%%%%%%%%%%%%
%%  Directories
\def\jpgdir{../img}
\def\imgdir{../img}

%%%%%%%%%%%%%%%%%%%%%%%%%%%%%%%%%%%%%%%%%%
%%%%%%%%%%%%%%%%%%%%%%%%%%%%%%%%%%%%%%%%%%
%% Version 5
%    Jan 3, 2013
%       Added \subhead, \problem, \numexamp
%    Feb 27, 2012
%       Added \xbar, \Var, \Cov  macros
%    Feb 11, 2012
%       Reorganized to use with header.tex and new mkpdf.sh
%       added some macros
%    Feb 5, 2012
%       Copied from 18.03
%       Added myhide code.
%       Removed topics codes
%%%%%%%%%%%%%%%%%%%%%%%%%%%%%%%%%%%%%%%%%%

%%  Macros
\def\partHeading#1#2{\mcent{\textrm{\large \bf Part #1 \hspace{2.3ex}#2}}}


%%  Simple macros
\def\mycomment#1{{\color{red}#1}}
\def\mcent#1{\hspace*{\stretch{1}}#1\hspace*{\stretch{1}}{ }}
\def\hs#1{\hspace*{#1 ex}}
\def\nl#1{\\[.#1ex]}
\def\ds{\displaystyle}
\def\cont{\vspace*{\stretch{1}}\emph{(continued)} \newpage}
\def\reading{\textbf{Reading. }}
\def\examples{\textbf{Examples. }}
\def\example{\textbf{Example. }}
\def\examp#1{\textbf{Example #1. }}
\def\problem{\textbf{Problem. }}
\def\problems{\textbf{Problems. }}
\def\subhead#1{\textbf{#1.}}
\def\indetop#1#2#3{\stackrel{\; \scriptstyle{{#1}{#3}{#2}}}{=} \;}
\def\indet#1#2{\stackrel{\scriptstyle{\; \frac{#1}{#2}}}{=} \;}
\def\indetzero{\indet{0}{0}}
\def\indetinf{\indetop{\infty}{\infty}{/}}
\def\qedbox{\rule{1ex}{1ex}}

\def\mybull{$\bullet$}
\def\W{\Omega}
\def\w{\omega}
\def\mycap{\,\cap\,}
\def\mycup{\,\cup\,}
\def\Norm{\text{N}}

\def\myref#1{(\ref*{#1})}

\def\mypart#1#2{\frac{\partial #1}{\partial #2}}
%      fully specified partial
\def\fp#1#2#3{(\partial #1/\partial #2)_{#3}}                %without \frac
\def\ffp#1#2#3{\left.\frac{\partial #1}{\partial #2}\right|_{#3}}  %with \frac
\def\mpp#1#2#3{\left(\mypart{#1}{#2}\right)_{#3}} 
\def\myderiv#1#2{\frac{d#1}{d#2}}
\def\mynderiv#1#2#3{\frac{d^{#3}#1}{d#2^{#3}}}
\def\mygrad{\boldsymbol{\nabla}}         %gradient
\def\gf#1#2{\left.\mygrad #1\right|_{#2}}
\def\endandindent{\\ \hspace*{20pt}}
\def\e#1{\mathrm{e}^{#1}}
\def\myIm{\textup{Im}}
\def\myRe{\textup{Re}}
\def\conj#1{\overline{#1}}
\def\tran{^\mathrm{T}}
\def\ans{{\bf \underline{answer:}}{ }}
\def\th{$^{\mathrm{th}}$}
\def\myhead#1{\noindent \textbf{#1}}
\def\tbf#1{\textbf{#1}}
\def\mysep{: }
\def\lap{{\mathcal{L}}}
\def\ilap{\lap^{-1}}
\def\myimply{\; \Rightarrow \;}
\def\myequiv{\; \Leftrightarrow \;}
\def\ft{\hat}
\def\xbar{\overline{X}}
\def\Var{\textup{Var}}
\def\Cov{\textup{Cov}}
\def\bypartshelp#1#2#3#4#5#6#7{\framebox{$\begin{array}[#7]{lll} 
#5=#1 & d#6=#2\\
d#5=#3 & #6=#4 \end{array}$}}
\def\byparts#1#2#3#4{\bypartshelp{#1}{#2}{#3}{#4}{u}{v}{l}}
\def\bypartst#1#2#3#4{\bypartshelp{#1}{#2}{#3}{#4}{u}{v}{t}}

%matlab 
\newcommand{\mlcmd}[2]{\textgreater{} \texttt{#1}\\ \texttt{#2}\\}
\newcommand{\mlcmdna}[1]{\textgreater{} \texttt{#1}\\}
\newcommand{\mlans}[2]{#1 =\\[.5ex]\hs3\parbox{5in}{#2}}
\def\mlcomm#1{{\color{blue}\% #1}\\}
\def\mlinstr#1{\mlcomm{#1}}
\def\mlcar{\^{ }}

%counters
\newcounter{examplectr}
\setcounter{examplectr}{1}
\renewcommand{\theexamplectr}{\arabic{examplectr}}
\newcommand{\numexamp}{\textbf{Example \arabic{examplectr}.{ }}\addtocounter{examplectr}{1}}

%matrices
\def\defleftbrace{(}
\def\defrightbrace{)}
\def\twobytwohelp#1#2#3#4#5#6#7{\left#6 \begin{array}{#1} #2 & #3 \\ #4 & #5 \end{array} \right#7}
\def\twobytwo#1#2#3#4{\twobytwohelp{rr}{#1}{#2}{#3}{#4}{\defleftbrace}{\defrightbrace}}
\def\twobytwoc#1#2#3#4{\twobytwohelp{cc}{#1}{#2}{#3}{#4}{\defleftbrace}{\defrightbrace}}
\def\twobytwodet#1#2#3#4{\twobytwohelp{rr}{#1}{#2}{#3}{#4}{|}{|}}
\def\twobytwodetc#1#2#3#4{\twobytwohelp{cc}{#1}{#2}{#3}{#4}{|}{|}}
\def\twobyonehelp#1#2#3{\left\defleftbrace \begin{array}{#1} #2\\ #3  \end{array} \right\defrightbrace}
\def\twobyone#1#2{\twobyonehelp{r}{#1}{#2}}
\def\twobyonec#1#2{\twobyonehelp{c}{#1}{#2}}
\def\threebythree#1#2#3#4#5#6#7#8#9{\left\defleftbrace \begin{array}{rrr} #1&#2&#3\\ #4&#5&#6\\ #7&#8&#9 \end{array} \right\defrightbrace}
\def\threebythreec#1#2#3#4#5#6#7#8#9{\left\defleftbrace \begin{array}{ccc} #1&#2&#3\\ #4&#5&#6\\ #7&#8&#9 \end{array} \right\defrightbrace}
\def\threebythreedet#1#2#3#4#5#6#7#8#9{\left| \begin{array}{rrr} #1&#2&#3\\ #4&#5&#6\\ #7&#8&#9 \end{array} \right|}
\def\threebythreedetc#1#2#3#4#5#6#7#8#9{\left| \begin{array}{ccc} #1&#2&#3\\ #4&#5&#6\\ #7&#8&#9 \end{array} \right|}
\def\threebyone#1#2#3{\left\defleftbrace \begin{array}{r} #1\\ #2\\ #3 \end{array} \right\defrightbrace}
\def\threebyonec#1#2#3{\left\defleftbrace \begin{array}{c} #1\\ #2\\ #3 \end{array} \right\defrightbrace}
\def\threebytwo#1#2#3#4#5#6{\left\defleftbrace \begin{array}{rr} #1&#2\\ #3&#4\\ #5&#6 \end{array} \right\defrightbrace}
\def\threebytwoc#1#2#3#4#5#6{\left\defleftbrace \begin{array}{cc} #1&#2\\ #3&#4\\ #5&#6 \end{array} \right\defrightbrace}

%Vectors and line segments
\def\vb#1{\mathbf{#1}}  %bold
\def\va#1{\overrightarrow{#1}}  %arraow
\def\vba#1{\overrightarrow{\mathbf{#1}}}  %bold/arraow
\def\vl#1{\overline{#1}} %overline
\def\vbl#1{\overline{\mathbf{#1}}} %bold/overline
\def\vu#1{\widehat{#1}} %unit vector
\def\vbu#1{\widehat{\mathbf{#1}}} %bold/unit vector
\def\un#1{\frac{#1}{\left|#1\right|}}  %normalize vector
\def\vbi{\vbu{i}}
\def\vbj{\vbu{j}}
\def\vbk{\vbu{k}}
\def\vc#1{\langle #1 \rangle}
\def\vcb#1{\left\langle #1 \right\rangle}

\newcommand{\st}[1]{\ensuremath{^{\scriptstyle \textrm{#1}}}}
\newcommand{\undertext}[1]{\ensuremath{\underline{\textrm{#1}}}}
\newcommand{\fracc}{\displaystyle\frac}
\newcommand{\summ}{\displaystyle\sum}

%%%%%%%%%%%%%%%%%%%%%%%%%%%%
%% tikz
\usetikz{\tikzset{
addarrow/.style={postaction={decorate},
        decoration={markings,mark=at position #1 with {\arrow{>}}}}
}}

%%%%%%%%%%%%%%%%%%%%%%%%%%%
%% Sample Complex formatting macros
%\makeatletter
%\renewcommand\section{\goodbreak\@startsection {section}{1}{\z@}%
%%%                                   {-3.5ex \@plus -1ex \@minus -.2ex}%
%                                   {-3.5ex \@plus -1ex \@minus -.2ex}%
%                                   {2.3ex \@plus.2ex}%
%                                   {\normalfont\large\bfseries%
%                                     \centering\sectionname\ }}

%\renewcommand\subsection{\goodbreak\@startsection{subsection}{2}{\z@}%
%%                                     {-3.25ex\@plus -1ex \@minus -.2ex}%
 %                                    {-2ex\@plus -1ex \@minus -.2ex}%
%%                                     {1.5ex \@plus .2ex}%
%                                     {1.25ex \@plus .2ex}%
%                                     {\normalfont\large\bfseries}}
%\makeatother

%\renewcommand{\thesection}{\Roman{section}}
%\newcommand\sectionname{Part}
%\newcommand{\alphalist}{% changes enumerate 1st level to (a)...(z)
%  \renewcommand{\theenumi}{\alph{enumi}}%
%  \renewcommand{\labelenumi}{\theenumi)}%
%}
%\alphalist
%%%%%%%%%%%%%%%%%%%%%%%%%%%%%%%%%%%%%%%%%%

\usepackage{ifthen} % needed for ifthenelse

\def\needstobedone{XXX***XXX}

\def\whichterm{Spring 2013}
%\def\whichprogram{ESG}
\def\teacher{Dr. Jeremy Orloff and Dr. Jonathan Bloom}
\def\email{jorloff}

%\def\website{web.mit.edu/jorloff/www/18.05}
%\def\matlabadd{\intentionalerror{need matlab dir}}
%\def\matlabdir{\intentionalerror{need matlab dir}}

\def\wheredue{in 2-285{}}
%\def\finalinfo{To be anounced}

\def\psdue#1{\textbf{Pset #1 due}}
%\def\exam#1#2{\textbf{Exam #1: covering #2}}
%\def\review{Discussion, review and catch up.}
%\def\reviewentry#1#2{\textbf{Continuation: } (#1) \, #2}
%\def\recitationentry#1#2{\textbf{Problem section:} (#1) \, #2}

%\def\examonecovers{\intentionalerror}
%\def\examtwocovers{\intentionalerror}

%Problem set due dates
\def\duetime{4:30 PM}
\def\psonedue{Monday, Feb. 11 at \duetime}
\def\pstwodue{Tuesday, Feb. 19 at \duetime}
\def\psthreedue{Monday, Feb. 25 at \duetime}
\def\psfourdue{Monday, Mar. 4 at \duetime}
\def\psfivedue{Monday, Mar. 18 at \duetime}
\def\pssixdue{Monday, Apr. 8 at \duetime}
\def\pssevendue{Wednesday, Apr. 17 at \duetime}
\def\pseightdue{Monday, Apr. 29 at \duetime}
\def\psninedue{Monday, May 6 at \duetime}
\def\pstendue{Not to be turned in}

\def\psetblurb{You are encouraged to discuss the problem sets with each other.
You must either list any collaborators or write "no collaborators" at the 
top of your paper.
You may \emph{not} read another student's paper for help with a problem.
You must write your solutions in your own words.}




%%%%%%%%%%%%%%%%%%%%%%%%%%%%%%%%%%%%%%%%%%

\newcommand{\mynull}[1]{}
\newcommand{\myquestA}[1]{#1}
\newcommand{\mysolA}[1]{#1}
% if hiding then print #1 else print #2
\newcommand{\myhidehide}[2]{#1}
\newcommand{\myhideshow}[2]{#2}

\def\myquest{\myquestA}
\def\mysol{\mysolA}
\def\myhide{\myhideshow}


\newcommand{\RR}{{\mathbb R}}
\newcommand{\bs}[1]{\hbox{\boldmath$#1$}}
\DeclareRobustCommand{\transp}{^{\rm T}}
\newcommand{\lsub}[1]{{\hbox{\footnotesize$#1$}}}
\newcommand{\lexp}[1]{{\hbox{\footnotesize$#1$}}}
\newcommand{\blankpagemark}{\centerline{\sl This page intentionally blank.}}
\DeclareRobustCommand{\prm}{^{\hbox{\footnotesize$\,\prime$}}}
\DeclareRobustCommand{\pprm}{^{\hbox{\footnotesize$\,\prime\prime$}}}
\newenvironment{m}[1]{\left[\begin{array}{@{\,}#1@{\,}}}{\end{array}\right]}
%\newcommand{\text}{\mbox}
\newcommand{\mat}[1]{\begin{bmatrix} #1 \end{bmatrix}}

%------------------------------------------------------------------------<

\begin{document}

\thispagestyle{empty}

%\hbox to \textwidth{\Large 18.06\hfill Professor
%  Strang \hfill Quiz 2\hfill April 11th, 2012\hfill}
%\vspace{3.5ex}

%%%%%%%%%%%%%%%%%%%%%%%%%%%%%%%%%%%%%%%%%%%%%%%%%%%%%%%%%%%%%%%%%%%%%%%%%%%%%
\begin{edXcourse}{18.05x}{18.05x Probability and Statistics}

\begin{edXchapter}{Unit 1}

\begin{edXsection}{Week 1}


\begin{edXsequential}

\newpage

\begin{edXproblem}{Problem 1}{40}

\begin{enumerate}

\item Find the projection $p$ of the vector $b$ onto the plane of $a_{1}$ and
$a_{2},$ when

%\begin{center}

\hspace{4em} $$  
b =
  \left [ \begin{array}{c}
1\\
0\\
0\\
1
\end{array} \right ],
\, \, \, \, \, \, \, \,
a_{1} = 
  \left [ \begin{array}{c}
1\\
7\\
1\\
7
\end{array} \right ], 
\, \, \, \, \, \, \, \,
a_{2} =  
  \left [ \begin{array}{r}
-1\\
7\\
1\\
-7
\end{array} \right ] .
$$

Please enter your answer as a row matrix, eg {\tt [1,2,3,4]}:	% edX

\edXabox{type='symbolic' size='90' expect='[4/25, 0, 0, 28/25]'}	% edX


\begin{sol}
Observe that $a_1^T a_2 = 0$. Thus
\begin{eqnarray*}
p = \frac{a_1^T b}{a_1^T a_1} a_1 +  \frac{a_2^T b}{a_2^T a_2} a_2 = \frac{8}{100}a_1 - \frac{8}{100} a_2 = \begin{bmatrix} 4/25 \\ 0 \\ 0 \\ 28/25 \end{bmatrix}.
\end{eqnarray*}
\end{sol}

\item What projection matrix $P$ will produce the projection 
$p=Pb$ for every vector $b$ in $\mathbb{R}^{4}\, ?$

\edXabox{type='symbolic' size='90' expect='[[2/100,0,0,14/100],[0,98/100,14/100,0],[0,14/100,2/100,0],[14/100,0,0,98/100]]' options='matrix'}	% edX

\begin{sol}
Let $A$ be the $4 \times 2$ matrix with columns $a_1$, $a_2$. $P$ is given by $P =
A(A^TA)^{-1}A^T$. Notice that
$$
A^T A = 
\begin{bmatrix}
100 & 0 \\
0 & 100
\end{bmatrix}.
$$
($a_1$ and $a_2$ are orthogonal and of same length.)

Thus 
$$
P = \frac{1}{100} AA^T = \frac{1}{100} 
\begin{bmatrix}
2 & 0 & 0 & 14 \\
0 & 98 & 14 & 0 \\
0 & 14 & 2 & 0 \\
14 & 0 & 0 & 98
\end{bmatrix}.
$$
\end{sol}

\newpage
\item What is the determinant of $I - P \, ?$ Explain your answer. 

\edXabox{type='string' expect='None'} % edX - this kind of problem has to be rewritten

\begin{sol}

$I-P$ is the matrix of the projection to the orthgonal complement of $C(A)$, i.e.~$N(A^T)$. In particular, $I-P$ has rank the dimension of $N(A^T)$, which is 3. Thus $I-P$ is singular, and $\det(I-P) = 0$. 
\end{sol}

\item \underline{What are all nonzero eigenvectors of $P$ with 
eigenvalue} $\lambda = 1 \, ?$

How is the number of independent eigenvectors
with $\lambda = 0$ of a square matrix $A$ connected to the rank of $A \, ?$ 
\hspace{2em}

\edXabox{type='string' expect='None'} % edX - this kind of problem has to be rewritten
 
(You could answer (c) and (d) even if you don't answer (b).)


\begin{sol}
The non-zero eigenvectors with eigenvalue $\lambda = 1$ are all the non-zero linear combinations of $a_1$ and $a_2$, i.e.~all the non-zero vectors in the plane spanned by $a_1$ and $a_2$.

Suppose $A$ is a $n \times n$ matrix, with rank $r$.
\begin{center}
\# independent zero-eigenvectors of $A$  =  \# independent vectors in $N(A)$

= dimension of $N(A)$  =  $n-r$
\end{center}
\end{sol}

\end{enumerate}

\end{edXproblem}
\newpage

\begin{edXproblem}{Problem 2}{30}

\begin{enumerate}
\item Suppose the matrix $A$ factors into $A=PLU$ with a permutation
matrix $P,$ and 1's on the diagonal of $L$ (lower triangular) and
pivots $d_{1} , \ldots , d_{n}$ on the diagonal of $U$ (upper triangular).


What is the determinant of $A \, ?$ \hspace{2em} EXPLAIN WHAT RULES YOU ARE USING.

\edXabox{type='string' expect='None'} % edX - this kind of problem has to be rewritten


\begin{sol} Use
$$
\det(A) = \det(P)\cdot \det(L)\cdot \det(U)
$$
where we make two uses of the rule $\det(MN) = \det(M) \det(N)$, for any two $n \times n$ matrices $M$ and $N$. We will compute each of the determinants on the right-hand side.

The determinant of a triangular matrix is the product of its diagonal entries; this is true whether the matrix is upper or lower triangular. Thus
$$ \det (L) = 1 \qquad \text{and} \qquad \det(U) = d_1 \cdot d_2 \cdot \ldots \cdot d_n.
$$
The determinant changes sign whenever two rows are swapped. Thus 
\begin{edXmath}
\det(P) = 
\begin{cases}
+1 & \text{if $P$ is even (even \# of row exchanges)} \\
-1 & \text{if $P$ is odd (odd \# of row exchanges)}
\end{cases}
\end{edXmath}
and so
$$ \det(A) = \pm d_1 \cdot d_2 \cdot \ldots \cdot d_n$$
where the sign depends on the parity of $P$.

\end{sol}

\item Suppose the first row of a new matrix $A$ consists of the numbers
$1,2,3,4$. Suppose the cofactors $C_{ij}$ of that first row are the
numbers $2,2,2,2$.

(Cofactors already include 
the $\pm$ signs.) 


\vspace{2ex}
Which entries of $A^{-1}$ does this tell you and what are those entries?

\edXabox{type='string' expect='None'} % edX - this kind of problem has to be rewritten

\begin{sol}
Using the cofactor expansion in the first row gives
\begin{eqnarray*}
\det(A) & = & a_{11} C_{11} + a_{12}C_{12} + a_{13}C_{13} + a_{14}C_{14} \\
& = & 1 \times 2 + 2 \times 2 + 3 \times 2 + 4 \times 2 \\
& = & 20
\end{eqnarray*}
As $A^{-1} = C^T /\det(A)$, where $C$ is the cofactor matrix, this data gives us the entries of the first column of $A^{-1}$; they are all $2 / 20 = 1/10$. 
\end{sol}

\newpage
\item What is the determinant of the matrix $M(x) \, ?$ For which values
of $x$ is the determinant equal to zero?

\begin{center}


\[
M(x) = 
  \left [ \begin{array}{crcl}
1 & 1 & 1 & 1\\
1 & -1 & 2 & x\\
1 & 1 & 4 & x^{2}\\
1 & -1 & 8 & x^{3}

\end{array} \right ]
\]


\end{center}

${\rm det}(M) = $ \edXabox{type='symbolic' size='90' expect=' -6 * x**3 +12*x**2 + 6*x - 12' } % edX

${\rm det}(M) = 0$ when $x=$ \edXabox{type='symbolic' options='matrix' expect='[1,-1,2]'} % edX

(enter your answer as a row vector)	% edX

\begin{sol}

\textbf{Solution no.~1.}

From, for instance, the `Big Formula', we know that $\det(M)$ is a cubic polynomial in $x$. Say
$$
\det(M) = a x^3 + b x^2 + cx +d.
$$
We can calculate $d$ by setting $x=0$. Using the cofactor expansion in the last column, we get that
$$
d = -
\begin{vmatrix}
1 & -1 & 2 \\
1 & 1 & 4 \\
1 & -1 & 8 
\end{vmatrix}
= -
\begin{vmatrix}
1 & -1 & 2 \\
0 & 2 & 2 \\
0  & 0 & 6
\end{vmatrix}
= -12.
$$

We will determine the other coefficients of $\det(M)$ by finding three roots for it. $x$ is a root of $\det(M)$ if and only if $M(x)$ is a singular matrix. Now, notice that 

\begin{center}
\begin{tabular}{rccl}
$(1, 1, 1)$ & = & $(x,x^2, x^3)$ &  for $x=1$ \\
$(1, -1, 1)$ & = & $(x,x^2, x^3)$ & for $x=-1$ \\
$(2, 4, 8 )$ & = & $(x,x^2, x^3)$ & for $x=2$. \\
\end{tabular}
\end{center}
Thus $M(x)$ is singular for $x=1, -1$ and $2$; moreover, this implies that $$\det(M) = a (x-1)(x+1)(x-2).$$
As $d = 2a$, we must have $a =-6$. Thus
$$
\det(M) =  -6(x-1)(x+1)(x-2) = -6 x^3 +12x^2 + 6x - 12.
$$
The values of $x$ for which $M(x)$ is singular are $1, -1$ and $2$.
\newline

\textbf{Solution no.~2.}
$$
\begin{vmatrix}
1 & 1 & 1 & 1\\
1 & -1 & 2 & x\\
1 & 1 & 4 & x^{2}\\
1 & -1 & 8 & x^{3}
\end{vmatrix}
=
\begin{vmatrix}
1& 1& 1 & 1 \\
0 & -2 & 1 & x-1 \\
0 & 0 & 3 & x^2 -1 \\
0 & -2 & 7  & x^3 -1
\end{vmatrix}
= 
\begin{vmatrix}
1& 1& 1 & 1 \\
0 & -2 & 1 & x-1 \\
0 & 0 & 3 & x^2 -1 \\
0 & 0 & 6 & x^3 -x
\end{vmatrix}
$$
$$
=
\begin{vmatrix}
 -2 & 1 & x-1 \\
0 & 3 & x^2 -1 \\
 0 & 6 & x^3 - x
\end{vmatrix}
=
-2
\begin{vmatrix}
3 & x^2 -1 \\
 6 & x^3 - x
\end{vmatrix}
= -6 x^3 +12x^2 + 6x - 12
$$
In the first step, subtract the first row from the second, third and fourth rows. 
In the second step, subtract the second row from the fourth. For the third and fourth steps, use the cofactor expansion in the first column.
\newline

We factorize $\det(M)$ by guessing roots, trying small integers; we find that $1, -1$ and $2$ are all roots, which gives
$$
\det(M) =  -6(x-1)(x+1)(x-2).
$$
The values of $x$ for which $M(x)$ is singular are $1, -1$ and $2$.
\end{sol}

\end{enumerate}

\end{edXproblem}
%%%%%%%%%% 	PROBLEM 3

\newpage

\begin{edXproblem}{Problem 3}{30}

\begin{enumerate}

\item Starting from independent vectors $a_{1}$ and $a_{2},$ use
Gram-Schmidt to find formulas for two orthonormal vectors $q_{1}$ and
$q_{2}$ (combinations of $a_{1}$ and $a_{2}$):

$q_1=$ \edXabox{type='symbolic' size='90' expect='a_1/||a_1||'} % edX

$q_2=$ \edXabox{type='symbolic' size='90' expect=' \frac{ a_2 - (a_2^T q_1)q_1}{|| a_2 - (a_2^T q_1)q_1||} '} % edX - broken


\begin{sol}

\[
q_{1}= \frac{a_1}{ ||a_1||}
\]


\[
q_{2}= \frac{ a_2 - (a_2^T q_1)q_1}{|| a_2 - (a_2^T q_1)q_1||}
= \big( a_2 - \frac{(a_2^T a_1)}{a_1^T a_1} a_1 \big)\, / \, || a_2 - \frac{(a_2^T a_1)}{a_1^T a_1} a_1 ||
\]


\end{sol}

\item The connection between the matrices $A = [a_{1} \, \, a_{2}]$ and
$Q= [q_{1} \, \, q_{2}]$ is often written $A=QR.$ From your answer to 
Part (a), what are the entries in this matrix $R \, ?$

$R=$ \edXabox{type='symbolic' size='90' expect='[[||a_1||,a_2^T q_1],[0,||a_2-(a_2^T q_1)*q_1]]'} % edX - broken

\begin{sol}
Re-arranging the expressions above gives
$$
a_1 = q_1 ||a_1||
$$
$$
a_2 = (a_2^T q_1) q_1 + ||a_2 - (a_2^T q_1) q_1|| q_2
$$
and thus
$$
R = 
\begin{bmatrix}
a_1^T q_1 & a_2^T q_1 \\
a_1^T q_2 & a_2^T q_2
\end{bmatrix}
=
\begin{bmatrix} 
||a_1|| & a_2^T q_1\\ 
0 &  ||a_2 - (a_2^T q_1) q_1|| 
\end{bmatrix}
$$

\end{sol}

\newpage
\item The least squares solution $\widehat{x}$ to the equation $Ax = b$ 
comes from solving what equation? If $A = QR$ as above, show that $R\,\widehat{x} = Q^{T}b.$

\begin{sol}
$\widehat{x}$ comes from solving $A^T A \widehat{x} = A^T b$. 

\edXabox{type='string' expect='None'} % edX - this kind of problem has to be rewritten
 
Suppose we have $A = QR$. Notice that:
\begin{itemize}
\item $Q^T Q = I$, so $A^T A = (QR)^T QR = R^T Q^T Q R = R^T R$.
\item As $a_1$ and $a_2$ are independent, $R$ is invertible. Thus $R^T$ is also invertible.
\end{itemize}
Thus we have
$$
\begin{matrix}
  & A^T A \, \widehat{x} & = & A^T b \\
\Leftrightarrow
 & R^T R \, \widehat{x}  &  = & R^T Q^T b \\
\Leftrightarrow  & R \, \widehat{x} &  = & Q^T b.
\end{matrix}
$$

\end{sol}

\end{enumerate}

\end{edXproblem}
%%%%%%%%%%%%%%%%%%%%%%%%%%%%%%%%%%%%%%%%%%%%%%%%%%%%%%%%%%%%%%%%%%%%%%%%%%%%%

\end{edXsequential}
\end{edXsection}
\end{edXchapter}
\end{edXcourse}

%%%%%%%%%%%%%%%%%%%%%%%%%%%%%%%%%%%%%%%%%%%%%%%%%%%%%%%%%%%%%%%%%%%%%%%%%%%%%

\end{document}




